\chapter{Preface}
\textbf{This Chapter will be removed eventually and only serves as a disclaimer and information page for the reader on the approach taken for this project}

As of now, these few pages are supposed to provide a broad overview of the writing style and approach taken for writing this thesis.

The goal is to write the thesis in parallel to the implementation of the tool, so whenever a new algorithm is analyzed it is first written about in the thesis here, then implemented and evaluated, followed by the documentation in this thesis.
\\\\
At the end, all the results will be aggregated and compared, to provide some final remarks and provide ideas on how to expand on the current tool
\\\\\\

Current Timetable/Milestones:\\
Mid-October:\\
-  Finish Base-Framework for the tool, to select planes like tables and detect objects above it, for this both the meta passthrough api aswell as the depth-api are going to be used (existing tools for selecting planes and getting a depth-map of the camera feed to detect objects on the selected surfaces)\\
- Best-Case -> Already black out the found objects, so its basically a black box covering the object, which is updated in real time, should be easy to achieve with the camera coordinates of the objects and a shader (Passthrough API allows Application of Shaders to the Camera Feed, so the concept is to implement the Inpainting as a shader)
\\\\
Mid-November: \\
- Finish at least 3 Methodology Evaluations, maybe optimize the tool a bit\\
- Trial of combining the tool with digital objects, Demo: Interactive Chess Board on a Table with a few notebooks and pens, maybe a glass of water to test reflection behavior
\\\\
Mid-December:\\
- Have the Chapters Fundamentals and Concept completely finished\\
- Best-Case very far with Evaluation
\\\\
End-December / Early January (Sometime during the Chrismas Holidays):\\
- Have a first draft to send in for a first review, then there should still be a few weeks time to work in the feedback



