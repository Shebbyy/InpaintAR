\chapter{Kurzfassung}

\begin{german}
Aufgrund der technologischen Fortschritte, sowohl in der Hardware, als auch in der Software-Integration, steigt die Beliebtheit von Augmented Reality mit jedem Jahr an.
\\
Ein sehr gängiges Problem in aktueller Augmented Reality Software ist die Ablenkung, welche durch physische Objekte, die um den digitalen Inhalten liegen. Beispielsweise bei der immersiven Analyse von Daten.
\\\\
In dieser Bachelorarbeit liegt der Fokus auf der Analyse verschiedener Methoden und Algorithmen zur Verdeckung dieser Objekte, wodurch eine aufgeräumte, saubere Umgebung für die digitalen Elemente geschaffen wird.
\\
Um dies zu erreichen wird ein Werkzeug entwickelt, welches erlaubt Flächen zu wählen, welche dann dazu verwendet werden, um Objekte auf diesen Flächen zu finden und darauffolgend durch die Nutzung von Inpainting zu verdecken. Mithilfe dieses Werkzeugs werden mehrere Methoden und Algorithmen sowohl auf Qualität, als auch auf Leistung evaluiert, um sowohl die Glaubbarkeit und Ordnung der generierten Umgebungen, während parallel dazu ein Fokus auf die Minimierung von Rechenresourcen und Laufzeit gelegt wird.
\\
Diese Ergebnisse werden zueinander verglichen, sodass Leser sowohl über die Vor- als auch Nachteile der jeweiligen Ansätze einen Überblick bekommen.
\end{german}